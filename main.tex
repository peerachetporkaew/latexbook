\documentclass[17pt,fancychapters]{report}
\usepackage[a4paper, total={6in, 8in}]{geometry}
\usepackage{listings}
\usepackage{color}
\usepackage{setspace}
\usepackage{hyperref}
\usepackage{amsmath}
\usepackage{amsthm}
\usepackage{graphicx}
\usepackage{geometry}
\usepackage{subcaption}
\usepackage{cancel}
\usepackage{tikz}
\usepackage{color}
\usepackage{titlesec}
\usepackage{hyperref}

\titleformat{\chapter}
{\normalfont\LARGE\bfseries}{บทที่ \thechapter}{1em}{}

\titleformat{\section}
{\normalfont\bfseries}{\thesection}{1em}{}

\renewcommand{\figurename}{ภาพที่ }
\renewcommand{\contentsname}{สารบัญ }

\usepackage{fontspec}
\usepackage{xltxtra}
\XeTeXlinebreaklocale "th_TH"
\XeTeXlinebreakskip = 0pt plus 1pt  
\usepackage{fonts-tlwg}

\defaultfontfeatures{Scale=1.6}

\setmainfont[
BoldFont={THSarabunNewBold.ttf},
ItalicFont={THSarabunNewItalic.ttf},
BoldItalicFont={THSarabunNewBoldItalic.ttf}
]{THSarabunNew.ttf}

\linespread{1.3}

\usetikzlibrary{calc,trees,positioning,arrows,chains,shapes.geometric,%
    decorations.pathreplacing,decorations.pathmorphing,shapes,%
    matrix,shapes.symbols}

\geometry{top=1.3in,bottom=1.3in}
\hypersetup{
    colorlinks,
    citecolor=black,
    filecolor=black,
    linkcolor=black,
    urlcolor=black
}

\definecolor{DarkerGreen}{RGB}{0,179,45}

\definecolor{Code}{rgb}{0,0,0}
\definecolor{Decorators}{rgb}{0.5,0.5,0.5}
\definecolor{Numbers}{rgb}{0.5,0,0}
\definecolor{MatchingBrackets}{rgb}{0.25,0.5,0.5}
\definecolor{Keywords}{rgb}{0,0,1}
\definecolor{self}{rgb}{0,0,0}
\definecolor{Strings}{rgb}{0,0.63,0}
\definecolor{Comments}{rgb}{0,0.63,1}
\definecolor{Backquotes}{rgb}{0,0,0}
\definecolor{Classname}{rgb}{0,0,0}
\definecolor{FunctionName}{rgb}{0,0,.7}
\definecolor{Operators}{rgb}{0,0,0}
\definecolor{Background}{rgb}{0.98,0.98,0.98}

\lstdefinestyle{python}{
  numbers=left,
  numberstyle=\footnotesize,
  numbersep=1em,
  xleftmargin=1em,
  framextopmargin=2em,
  framexbottommargin=2em,
  showspaces=false,
  showtabs=false,
  showstringspaces=false,
  frame=l,
  tabsize=4,
  % Basic
  basicstyle=\ttfamily\small\setstretch{1},
  backgroundcolor=\color{Background},
  language=Python,
  % Comments
  commentstyle=\color{Comments}\slshape,
  % Strings
  stringstyle=\color{Strings},
  morecomment=[s][\color{Strings}]{"""}{"""},
  morecomment=[s][\color{Strings}]{'''}{'''},
  % keywords
  morekeywords={import,from,class,def,for,while,if,is,in,elif,else,not,and,or,print,break,continue,return,True,False,None,access,as,,del,except,exec,finally,global,import,lambda,pass,print,raise,try,assert},
  keywordstyle={\color{Keywords}\bfseries},
  % additional keywords
  morekeywords={[2]@invariant},
  keywordstyle={[2]\color{Decorators}\slshape},
  emph={self},
  emphstyle={\color{self}\slshape},
  breaklines=true
}

\newtheorem{exmp}{Example}[section]

\newcommand{\codeExample}[2]{
	\begin{exmp}
      #1
      \noindent\begin{minipage}{\linewidth}
      \begin{lstlisting}[style=python]
          #2
      \end{lstlisting}
      \end{minipage}
    \end{exmp}
}

\newcommand\MyLBrace[2]{%
  \left.\rule{0pt}{#1}\right\}\text{#2}}
  
\tikzset{
>=stealth',
  punktchain/.style={
    rectangle, 
    rounded corners, 
    % fill=black!10,
    draw=black, very thick,
    text width=10em, 
    minimum height=3em, 
    text centered, 
    on chain},
  line/.style={draw, thick, <-},
  element/.style={
    tape,
    top color=white,
    bottom color=blue!50!black!60!,
    minimum width=8em,
    draw=blue!40!black!90, very thick,
    text width=10em, 
    minimum height=3.5em, 
    text centered, 
    on chain},
  every join/.style={->, thick,shorten >=1pt},
  decoration={brace},
  tuborg/.style={decorate},
  tubnode/.style={midway, right=2pt},
}

\title{สอนเขียนโปรแกรมด้วยภาษาไพธอนแบบพี่สอนน้อง}
\author{พีรเชษฐ  ปอแก้ว}
\date{December 30, 2015}

\begin{document}

\maketitle
\pagenumbering{gobble}
\newpage
\pagenumbering{roman}
\tableofcontents
\newpage
\pagenumbering{arabic}

\chapter{พื้นฐานภาษาไพธอนและการเขียนโปรแกรม}

\quad ภาษาไพธอนเป็นภาษาคอมพิวเตอร์ที่ได้รับความนิยมสูง มีเครื่องมือหรือไลบรารี่ให้เลือกใช้หลากหลาย ตั้งแต่ไลบรารี่พื้นฐานไปจนถึงการคำนวณทางคณิตศาสตร์ขั้นสูง หรือแม้แต่การสร้างภาพกราฟิกและเกมสามมิติ เมื่อเทียบกับภาษาอื่นๆ ภาษาไพธอน ถือว่าเป็นภาษาที่มองแล้วดู "สะอาดตา" เพราะไม่มีวงเล็บปีกกาเปิดปิดระหว่างบล็อคคำสั่ง อีกเหตุผลหนึ่ง คือ เมื่อเทียบกับภาษาอื่นๆ อย่างภาษาซีหรือจาวา หากใช้พัฒนาโปรแกรมแบบเดียวกันภาษาไพธอนเขียนคำสั่งสั้นกว่ามาก ทำให้การดีบัก แก้ไข และบำรุงรักษาระบบทำได้ง่ายกว่า

ภาษาคอมพิวเตอร์เป็นภาษาที่มีรูปแบบและโครงสร้างที่เป็นแบบแผน คอมพิวเตอร์จะนำเอาคำสั่งทีละคำสั่งไปประมวลผลและสร้างผลลัพธ์ทีละขั้นตอนต่อๆ กันเป็นกระบวนการทำงานหนึ่งๆ ตัวอย่างโปรแกรม 1-1 ทำงานโดยการรับค่าจากผู้ใช้เพื่อนแสดงผลลัพธ์ทักทายผู้ใช้ 

ทิป : ผู้อ่านสามารถลองเล่นภาษาไพธอนได้ผ่านทางเว็บเบราเซอร์ โดยใช้งานผ่านเว็บไซต์ https://repl.it/langauges/python3

\paragraph{โปรแกรม 1-1} สวัสดีชาวโลก

\begin{lstlisting}[style=python]
name = input("Enter your name :")
print("Hello, " + name)
\end{lstlisting}

\begin{figure}[h!]
	\center
	\includegraphics[width=450pt]{pic/helloworld.jpg}
	\caption{ผลลัพธ์ที่ได้จากโปรแกรม 1-1}
	\label{fig:example}
\end{figure}

บรรทัดที่ 1 โปรแกรมจะเรียกคำสั่ง input โดยแสดงข้อความว่า Enter your name : และหยุดรอรับคำสั่งจากผู้ใช้งาน เมื่อเราป้อนชื่อของเราลงไปแล้ว ชื่อของเราจะถูกนำไปเก็บไว้ในตัวแปรที่ชื่อ name (ชื่อตัวแปรจะตั้งชื่ออย่างไรก็ได้ ขอเพียงให้ไม่ซ้ำกับชื่อคำสั่งควบคุมของไพธอน) จากนั้นคำสั่งในบรรทัดที่ 2 ก็จะแสดงผลลัพธ์ออกมาว่า Hello, (ค่าที่ป้อน)

การประกาศตัวแปรภาษาไพธอนนั้นไม่มีการประกาศชนิดตัวแปรล่วงหน้า ทำให้ตัวแปรสามารถเก็บค่าชนิดต่างๆ ได้ทันที 
อย่างในตัวอย่างข้างต้นนี้ name คือ ตัวแปรที่เก็บค่าสายอักขระที่ผู้ใช้ป้อนเข้ามา คำสั่ง input คือ คำสั่งที่รับค่าจากผู้ใช้ และทำให้ตัวมันเองมีค่าเท่ากับค่าที่ผู้ใช้ป้อนเข้ามา ดังนั้นเมื่อรันคำสั่งแรกไปแล้ว name จึงมีค่าเท่ากับสิ่งที่ผู้ใช้ป้อนนั่นเอง

เราสามารถกำหนดค่าให้กับตัวแปรได้หลากหลายรูปแบบ ทั้งจำนวนเต็ม จำนวนทศนิยม นิพจน์(จริง,เท็จ) ข้อความ เช่น

\begin{lstlisting}[style=python]
x = 15
y = 10
z = x + y
real = True
not_real = False
message = "House"
print(message,z,real,not_real) #House 25 True False
\end{lstlisting}

แนะนำอ่านเพิ่มเติมที่ https://docs.python.org/3/library/stdtypes.html

สัญลักษณ์ \# ในบรรทัดที่ 7 คือการทำเครื่องหมายว่าข้อความหลังจากนี้ไม่ใช่ชุดคำสั่ง แต่เป็นหมายเหตุเหมือนโน๊ตสั้นๆ ว่าผลลัพธ์หรือขั้นตอนการทำงานนี้เป็นอย่างไร เราเรียกสั้นๆ ว่าส่วนข้อคิดเห็น หรือ Comment 

เนื้อหาต่อจากนี้จะกล่าวถึงคำสั่งโครงสร้างโปรแกรม ได้แก่ if..else, for, while, ตัวแปรแบบ list และ และการสร้างฟังก์ชั่น รับรองว่าเนื้อหาของเราไม่เหมือนในตำราเล่มไหนๆ แน่นอน เพราะเราเชื่อว่าการจะเขียนโปรแกรมให้ดีนั้นไม่ได้อยู่ที่ว่าคุณรู้คำสั่งมากแค่ไหน แต่ขึ้นอยู่กับว่าคุณได้อ่านโค้ดที่มีคุณภาพมากน้อยแค่ไหนต่างหาก ดังนั้นหลังจากที่เราเรียนรู้คำสั่งควบคุมพื้นฐานเหล่านี้แล้ว เราจะมาเขียนโปรแกรมจริงๆ เพื่อแก้ปัญหาจริงๆ ไปพร้อมๆ กัน

\section{if..else}

\paragraph{โปรแกรม 1-2} พื้นฐาน if...else
\begin{lstlisting}[style=python]
x1 = input("Enter an integer for X :")
y1 = input("Enter an integer for Y :")
x2 = int(x1)
y2 = int(y1)
if y2 > x2:
	print("Y > X")
else:
	print("X >= Y")
\end{lstlisting}

คำสั่ง int(x1) มีไว้แปลงค่าจากตัวอักษรให้เป็นตัวเลข เนื่องจากค่าที่รับมาจาก คำสั่ง input จะเป็นตัวอักษร ดังนั้นหากต้องการดำเนินการทางคณิตศาสตร์จะต้องแปลงค่าให้กลับมาเป็น integer (จำนวนเต็ม) เสียก่อน จะเห็นว่าคำสั่ง print ในบรรทัดที่ 6 และ 8 จะถูกเลือกทำตามเงื่อนไข y2 > x2 ถ้าเป็นจริงจะทำบรรทัดที่ 6 ถ้าเท็จข้ามไปทำบรรทัดที่ 8

ทิป : x1, y1, x2, y2 เป็นตัวแปรที่ตั้งชื่อขึ้นมาเอง ไม่ได้เป็นคำสั่งของภาษาไพธอนแต่อย่างใด เรามีอิสระในการปรับเปลี่ยนตัวแปรเหล่านี้ตราบใดที่ไม่ผิดกฏการตั้งชื่อตัวแปรของไพธอน

กรณีที่เรามีเงื่อนไขมากกว่า 1 เงื่อนไขสามารถเขียนเป็นคำสั่งได้ตามตัวอย่างที่ 1-3
\paragraph{โปรแกรม 1-3} พื้นฐาน if...elif...else
\begin{lstlisting}[style=python]
x1 = input("Enter an integer for X :")
y1 = input("Enter an integer for Y :")
x2 = int(x1)
y2 = int(y1)
if x2 == y2:
	print("X = Y")
elif x2 > y2:
	print("X > Y")
else:
	print("X < Y")
\end{lstlisting}

\section{for}

\begin{equation}
y = x^2 + z^2
\end{equation}
\section{while}


\chapter{เอกสารอ้างอิง}

\noindent Python Documentation \\
\url{https://docs.python.org/3/}

\hfill 

\noindent โจทย์พื้นฐานสำหรับการเขียนโปรแกรมภาษาซี \\
\url{https://www.w3resource.com/c-programming-exercises/}

\hfill

\noindent โจทย์คอมพิวเตอร์โอลิมปิกระดับประเทศ (สอวน) \\
\url{https://www.posn.or.th/?p=1055}

\hfill

\noindent Beginner Programming Project \\
\url{https://www.reddit.com/r/learnprogramming/comments/2a9ygh/1000_beginner_programming_projects_xpost/}







\end{document}